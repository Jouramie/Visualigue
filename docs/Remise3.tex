\documentclass[ULlof]{ULrapport}

% Chargement des packages
\usepackage[utf8]{inputenc}
\usepackage[autolanguage]{numprint}
\usepackage{icomma}
\usepackage{multirow}
\usepackage{tabularx}
\usepackage{pbox}
\usepackage{float}
\usepackage{pifont}

\setcounter{tocdepth}{2}

% Definition d'une commande pour presenter des cellules multilignes dans un tableau
%\newcommand{\cellulemultiligne}[1]{\begin{tabular}{@{}c@{}}#1\end{tabular}}

% Definition de colonnes en mode paragraphe avec alignement ajustable
% Cette definition requiert le chargement du package "array"
%    - alignement horizontal, parametre #1 : - \raggedright (aligne a gauche)
%                                            - \centering (centre)
%                                            - \raggedleft (aligne a droite)
%    - alignement vertical, parametre #2 : - p (aligne en haut)
%                                          - m (centre)
%                                          - b (aligne en bas)
%    - largeur, parametre #3 : longueur
%\newcolumntype{Z}[3]{>{#1\hspace{0pt}\arraybackslash}#2{#3}}

% Page titre
\TitreProjet{Tableau de hockey interactif}
\TitreRapport{Document de conception -- Remise 3}
\Destinataire{Martin Savoie}
\NomEquipe{The Javangers}
\TableauMembres{
   111\,127\,868  & Jérémie Bolduc & \\\hline
   111\,126\,228  & Simon-Pierre Deschênes & \\\hline
   111\,121\,082  & Émile Grégoire & \\\hline
   111\,130\,693  & Alexandre McCune & \\\hline
}
\DateRemise{27 novembre 2016}


\HistoriqueVersions{
   0.0 & 19 septembre 2016 & Création du document \\\hline
   0.1 & 2 octobre 2016 & Première version après la phase d'inception et le début de l'élaboration \\\hline
   1.0 & 6 novembre 2016 & Version pour la remise 2 \\\hline
   2.0 & 27 novembre 2016 & Version pour la remise 3 \\\hline
}

\begin{document}

\chapter{Diagrammes d'états}

Cette section présente les deux diagrammes d'états demandés pour l'application.

\begin{figure}[H]
	\centering
	\includegraphics[width=\textwidth]{{"fig/diagrams3/Diagramme etats joueur"}.png}
	\caption{Diagramme d'états pour un joueur}
\end{figure}

\begin{figure}[H]
	\centering
	\includegraphics[width=\textwidth]{{"fig/diagrams3/Diagramme etats rondelle"}.png}
	\caption{Diagramme d'états pour la rondelle}
\end{figure}

\chapter{Diagramme d'activité}

Cette section présente le diagramme d'activité pour la création d'une stratégie en mode image par image.

\begin{figure}[H]
	\centering
	\includegraphics[width=4in]{{"fig/diagrams3/Diagramme activite"}.png}
	\caption{Diagramme d'activité pour la création d'une stratégie image par image}
\end{figure}

\chapter{Diagramme de classe}

Cette section présente le diagramme de classe mis à jour pour le livrable 3.

\begin{figure}[H]
	\centering
	\includegraphics[width=\textwidth]{{"fig/diagrams3/Diagramme de classe"}.png}
	\caption{Diagramme de classe de conception avec packages}
\end{figure}

\end{document}
