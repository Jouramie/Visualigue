\documentclass[ULlof,ULlot]{ULrapport}

% Chargement des packages
\usepackage[utf8]{inputenc}
\usepackage[autolanguage]{numprint}
\usepackage{icomma}
\usepackage{multirow}
\usepackage{tabularx}
\usepackage{pbox}
\usepackage{float}
\usepackage{pifont}

\setcounter{tocdepth}{2}

% Definition d'une commande pour presenter des cellules multilignes dans un tableau
%\newcommand{\cellulemultiligne}[1]{\begin{tabular}{@{}c@{}}#1\end{tabular}}

% Definition de colonnes en mode paragraphe avec alignement ajustable
% Cette definition requiert le chargement du package "array"
%    - alignement horizontal, parametre #1 : - \raggedright (aligne a gauche)
%                                            - \centering (centre)
%                                            - \raggedleft (aligne a droite)
%    - alignement vertical, parametre #2 : - p (aligne en haut)
%                                          - m (centre)
%                                          - b (aligne en bas)
%    - largeur, parametre #3 : longueur
%\newcolumntype{Z}[3]{>{#1\hspace{0pt}\arraybackslash}#2{#3}}

% Page titre
\TitreProjet{Tableau de hockey interactif}
\TitreRapport{Document de conception}
\Destinataire{Martin Savoie}
\NomEquipe{The Javangers}
\TableauMembres{
   111\,127\,868  & Jérémie Bolduc & \\\hline
   111\,126\,228  & Simon-Pierre Deschênes & \\\hline
   111\,121\,082  & Émile Grégoire & \\\hline
   111\,130\,693  & Alexandre McCune & \\\hline
}
\DateRemise{2 octobre 2016}


\HistoriqueVersions{
   0.0 & 19 septembre 2016 & Création du document \\\hline
}

\begin{document}
% Chapitres
\chapter*{Introduction}
\label{s:intro}

Lorem ipsum dolor sit amet, consectetur adipiscing elit. Curabitur odio nisl, feugiat quis quam non, consectetur tempus leo. Etiam nec enim lacus. In porta tempor nisi. Aenean fermentum, sapien at tincidunt pharetra, nibh nunc vehicula urna, sed scelerisque elit risus at ex. Donec egestas, turpis a pellentesque posuere, nibh tellus malesuada elit, sit amet porttitor enim eros a felis. Integer non congue enim. Donec bibendum ex id elementum rutrum. Donec porta nunc et odio gravida, vel vehicula orci aliquet.

Pellentesque gravida fermentum lectus, in laoreet sapien facilisis laoreet. Nunc sit amet leo volutpat, ornare lorem ut, hendrerit ex. Donec lectus augue, interdum in placerat in, dignissim dictum diam. Mauris tincidunt leo nisl, eu convallis odio consectetur id. Vestibulum placerat sem non mattis convallis. Etiam quis lorem imperdiet, gravida felis a, venenatis justo. Praesent eu lorem diam. Phasellus purus mi, tincidunt quis sapien iaculis, eleifend hendrerit est. Sed in justo efficitur, vulputate massa nec, rhoncus tortor. Cras risus nisl, finibus non felis vitae, malesuada sollicitudin ex. Donec finibus sit amet nisi at condimentum. Vivamus vitae libero semper, iaculis orci eget, porttitor sem. Phasellus eget hendrerit mi. Ut feugiat, nulla eu pretium egestas, dui est pretium eros, et tristique ligula magna sed purus
\chapter{Vision}
\label{s:vision}

Cette section présente la vision du projet VisuaLigue, un outil de communication de stratégies sportives numérique. La lecture de cette partie est fortement recommandée pour un néophyte du projet, car elle présente le positionnement du produit, une description générale du logiciel et un sommaire des fonctionnalités. Ainsi, la lecture de cette section permet une compréhension globale du projet et une meilleure orientation dans son développement.

\section{Positionnement}
\subsection{Énoncé du problème}
Présentement, la majorité des entraineurs utilisent un tableau blanc avec un arrière-plan de patinoire pour dessiner et expliquer les stratégies. Or, les esquisses sont perdues après chaque entraînement et la visualisation des dessins n'est pas toujours simple.

De plus, le partage et le répertoriage des stratégies sont actuellement difficiles. Généralement, la façon de faire est de photographier le tableau blanc, puis d'envoyer l'image par courriel aux membres intéressés, ou encore de refaire le dessin sur papier ou sur une tablette graphique pour la répertorier.

L'autre problème majeur est la visualisation des stratégies. Souvent, lorsqu'un joueur est présent à une partie, il assiste à l'élaboration de la stratégie et comprend mieux le dessin. Généralement, le mouvement des joueurs est dessiné en temps réel et la gestuelle de l'entraineur est primordiale pour la compréhension. De plus, les trajectoires des joueurs sont souvent effacées pour libérer de l'espace. Or, pour un joueur qui était absent à l'entraînement ou pour un parent qui n'était pas sur la glace lors de l'élaboration de la stratégie, il est souvent difficile de comprendre la stratégie seulement avec le dessin.

\subsection{Opportunité}
Le produit est né d'une demande d'un entraineur de hockey junior qui se plaignait des méthodes traditionnelles d'esquisses de stratégies.

Non seulement un entraineur a manifesté un enthousiasme pour le produit, mais l'Association des entraineurs mineurs du Québec (AEMQ) est aussi intéressée par le produit. D'ailleurs, les spécifications actuelles ont été établies en collaboration avec le président de l'AEMQ.

Outre le milieu du hockey junior, le produit est facilement extensible au milieu professionnel. On remarque notamment l'usage des tableaux blancs dans le domaine professionnel du hockey. De plus, en rendant le logiciel suffisamment flexible, il serait possible d'étendre l'idée pour d'autres sports, notamment le soccer, le football, le volleyball, l'ultimate frisbee, le handball, le kinball, le curling et bien d'autres.

Ce projet pourrait donc avoir des répercussions sur un vaste éventail de sports, dont certaines ligues professionnelles qui engrangent des quantités impressionnantes d'argent. Plusieurs de ces sports sont présents un peu partout dans le monde, tant au niveau amateur que professionnel. La démarcation du produit pourrait d'ailleurs se faire à de nombreux événements sportifs tels que des tournois, des championnats internationaux et même les Jeux olympiques.

\subsection{Alternatives et compétition}
Les alternatives présentement sur le marché représentent davantage des outils de dessin sur ordinateur. Notamment, le logiciel ConceptDraw PRO\footnote{\url{http://www.conceptdraw.com/How-To-Guide/ice-hockey-diagram-defensive-strategy-neutral-zone-trap}} offre des extensions pour le dessin de stratégies de hockey.

De nombreux logiciels sont disponibles sur les tablettes pour le dessin. Certains de ces logiciels permettent la projection simultanée pour une meilleure visualisation par les joueurs.

Or, ces solutions permettent seulement de réaliser des images statiques. Celles-ci sont plus faciles à répertorier considérant leur support numérique. Toutefois, le problème de visualisation est toujours présent. Il est difficile de visualiser la stratégie à partir d'une image fixe.

\subsection{Résumé du produit}
VisuaLigue est une application qui permet la création et la visualisation de stratégies sportives. Un entraineur peut donc facilement placer les joueurs, les obstacles et les objets sur un terrain virtuel. Ces éléments peuvent ensuite être modifiés facilement grâce à une interface utilisateur conviviale. Le logiciel permet aussi de visualiser la stratégie de manière dynamique. Ainsi, l'entraineur peut démarrer la visualisation et montrer à tout le monde le déplacement des joueurs en temps réel. Il peut aussi mettre la visualisation sur pause, avancer, reculer et regarder image par image. Finalement, l'application permet la sauvegarde des jeux pour permettre le partage et le répertoriage.

Le tableau \ref{t:fonctionnalites} présente un sommaire non exhaustif des fonctionnalités du logiciel ainsi que les avantages offerts pour les parties prenantes.

\begin{table}[h]
	\begin{tabular}{|p{6cm}|p{8cm}|}
		\hline
		\bfseries{Fonctionnalité}              & \bfseries{Avantage pour les parties prenantes} \\\hline
		Création de stratégies numériquement   & Facile à partager, schéma plus clair, notation standardisée \\\hline
		Visualisation des stratégies (lecture, pause, avancer, reculer, image par image, etc.) & Meilleure compréhension des joueurs, surtout s'ils n'étaient pas présents lors de l'élaboration de la stratégie \\\hline
		Création de nouveaux types de sports   & Plus grande flexibilité pour les entraineurs. Extension du projet vers des sports autres que le hockey \\\hline
		Création de nouveaux types d'obstacles & Flexibilité du logiciel pour différents types d'entraînements \\
		\hline
	\end{tabular}
	\caption{Fonctionnalités et avantages pour les parties prenantes}
	\label{t:fonctionnalites}
\end{table}

D'autres fonctionnalités plus techniques ont été énoncées durant les discussions. Les points suivants ont notamment été soulevés:
\begin{itemize}
	\item Fonctionnalité d'annuler/rétablir
	\item Exporter les stratégies sous un format d'image (PNG, JPEG, etc.)
	\item Zoom
	\item Affichage des coordonnées de la souris lors du déplacement sur l'aire de jeu
	\item Option pour montrer/cacher le rôle des joueurs
\end{itemize}

Une liste exhaustive des fonctionnalités sera détaillée plus loin dans ce rapport.

\chapter{Modèle des cas d'utilisation}
\label{s:use_cases}
\newpage
\begin{flushleft}
	\textbf{Cas d'utilisation 1 : Créer une stratégie}\\
\end{flushleft}
\begin{tabular}{|p{16cm}|}
	\hline
	\\
	\textbf{Projet :} Visualigue\\
	\\
	\textbf{Niveau :} But d'utilisateur\\
	\\
	\textbf{Acteurs primaires :} Entraineur\\
	\\
	\textbf{Figurants et intérêts :} \\
	- Entraineur: Veut pouvoir créer des fichiers qui contiendront éventuellement des stratégies.\\
	\\
	\textbf{Postconditions :} L'entraineur aura un fichier qui pourra être utilisé pour élaborer une stratégie.\\
	\\
	\textbf{Principal scénario de succès :}\\
	1. Entraineur démarre le processus de création d'une stratégie.\\
	2. Système demande les informations en lien avec la nouvelle stratégie.\\
	3. Entraineur fourni à Système les informations nécessaires.\\
	4. Système demande l'endroit où le fichier devra être enregistré ainsi que le nom de ce dernier.\\
	5. Entraineur choisi l'endroit où le fichier devra être enregistré ainsi que son nom.\\
	6. Système crée le fichier et il est possible de le modifier.\\
	\\
	\textbf{Extensions :}\\
	*a. Entraineur annule le processus de création d'une stratégie.\\
	\hspace{1cm}1. Système demande à Entraineur s'il est certain de vouloir annuler la création d'une stratégie.\\
	\hspace{1cm}2. Entraineur confirme qu'il veut bien annuler la création d'une stratégie.\\
	\hspace{2cm}2a. Entraineur indique qu'il ne veut plus annuler la création d'une stratégie.\\
	\hspace{3cm}1. Système revient où il était dans le processus de création d'une stratégie.\\
	\hspace{1cm}3. Système retourne à la page où il se trouvait avant que le processus de création d'une stratégie ne soit démarré.\\
	1a. Entraineur était au milieu de l'édition d'un fichier.\\
	\hspace{1cm}1. Système demande à Entraineur s'il veut sauvegarder le fichier dont l'édition était en cours.\\
	\hspace{1cm}2. Entraineur choisi s'il veut sauvegarder ou non le fichier dont l'édition était en cours.\\
	5a. Entraineur entre un nom de fichier invalide.\\
	\hspace{1cm}1. Système informe Entraineur que le nom entré est invalide.\\
	\hspace{1cm}2. Entraineur entre un nom valide pour le fichier.\\
	5b. Entraineur choisi un emplacement invalide pour le fichier.\\
	\hspace{1cm}1. Système informe Entraineur que l'emplacement choisi est invalide.\\
	\hspace{1cm}2. Entraineur choisi un emplacement valide pour le fichier.\\
	\\
	\textbf{Fréquence d'occurrence :} Régulièrement\\
	\\
	\textbf{Questions ouvertes :} Quelles seront les informations en lien avec la nouvelle stratégie que l'entraineur devra entrer?\\
	\\
	\hline
\end{tabular}
\newpage
\begin{flushleft}
	\textbf{Cas d'utilisation 2 : Sauvegarder une stratégie}\\
\end{flushleft}
\begin{tabular}{|p{16cm}|}
	\hline
	\\
	\textbf{Projet :} Visualigue\\
	\\
	\textbf{Niveau :} But d'utilisateur\\
	\\
	\textbf{Acteurs primaires :} Entraineur\\
	\\
	\textbf{Figurants et intérêts :} \\
	- Entraineur: Veut pouvoir sauvegarder une stratégie qu'il a élaborée.\\
	\\
	\textbf{Postconditions :} La stratégie est sauvegardée et peut être chargée lors d'une prochaine utilisation du logiciel.\\
	\\
	\textbf{Principal scénario de succès :}\\
	1. Entraineur démarre le processus de sauvegarde de la stratégie.\\
	2. Système sauvegarde la stratégie dans le fichier de stratégie.\\
	\\
	\textbf{Extensions :}\\
	*a. Entraineur annule le processus de sauvegarde d'une stratégie.\\
	\hspace{1cm}1. Système demande à Entraineur s'il est certain de vouloir annuler la sauvegarde d'une stratégie.\\
	\hspace{1cm}2. Entraineur confirme qu'il veut bien annuler la sauvegarde d'une stratégie.\\
	\hspace{2cm}2a. Entraineur indique qu'il ne veut plus annuler la sauvegarde d'une stratégie.\\
	\hspace{3cm}1. Système revient où il était dans le processus de sauvegarde d'une stratégie.\\
	\hspace{1cm}3. Système retourne à la page où il se trouvait avant que le processus de sauvegarde d'une stratégie soit démarré.\\
	\\
	\textbf{Fréquence d'occurrence :} Régulièrement\\
	\\
	\hline
\end{tabular}
\newpage
\begin{flushleft}
	\textbf{Cas d'utilisation 3 : Charger une stratégie}\\
\end{flushleft}
\begin{tabular}{|p{16cm}|}
	\hline
	\\
	\textbf{Projet :} Visualigue\\
	\\
	\textbf{Niveau :} But d'utilisateur\\
	\\
	\textbf{Acteurs primaires :} Entraineur et Joueur\\
	\\
	\textbf{Figurants et intérêts :} \\
	- Entraineur: Veut pouvoir charger une stratégie qu'il a élaborée.\\
	- Joueur: Veut pouvoir charger une stratégie élaborée par l'entraineur.\\
	\\
	\textbf{Postconditions :} La stratégie est chargée et elle peut être modifiée ou visualisée.\\
	\\
	\textbf{Principal scénario de succès :}\\
	1. Entraineur ou Joueur démarre le processus de chargement de la stratégie.\\
	2. Système demande lequel des fichiers de stratégie charger.\\
	3. Entraineur ou Joueur choisi lequel des fichiers de stratégie charger.\\
	4. Système charge le fichier.\\
	\\
	\textbf{Extensions :}\\
	*a. Entraineur ou Joueur annule le processus de chargement d'une stratégie.\\
	\hspace{1cm}1. Système demande à Entraineur ou Joueur s'il est certain de vouloir annuler le chargement d'une stratégie.\\
	\hspace{1cm}2. Entraineur ou Joueur confirme qu'il veut bien annuler le chargement d'une stratégie.\\
	\hspace{2cm}2a. Entraineur ou Joueur indique qu'il ne veut plus annuler le chargement d'une stratégie.\\
	\hspace{3cm}1. Système revient où il était dans le processus de chargement d'une stratégie.\\
	\hspace{1cm}3. Système retourne à la page où il se trouvait avant que le processus de chargement d'une stratégie ne soit démarré.\\
	1a. Entraineur était au milieu de l'édition d'un fichier.\\
	\hspace{1cm}1. Système demande à Entraineur s'il veut sauvegarder le fichier dont l'édition était en cours.\\
	\hspace{1cm}2. Entraineur choisi s'il veut sauvegarder ou non le fichier dont l'édition était en cours.\\
	3a. Entraineur ou Joueur choisi un fichier invalide.\\
	\hspace{1cm}1. Système informe Entraineur ou Joueur que le fichier choisi est invalide.\\
	\hspace{1cm}2. Entraineur ou Joueur choisi un fichier valide.\\
	\\
	\textbf{Fréquence d'occurrence :} Régulièrement\\
	\\
	\hline
\end{tabular}

\newpage
\begin{flushleft}
	\textbf{Cas d'utilisation 4 : Visualiser une stratégie}\\
\end{flushleft}
\begin{tabular}{|p{16cm}|}
	\hline
	\\
	\textbf{Projet :} Visualigue\\
	\\
	\textbf{Niveau :} But d'utilisateur\\
	\\
	\textbf{Acteurs primaires :} Entraineur et Joueur\\
	\\
	\textbf{Figurants et intérêts :} \\
	- Entraineur: Veut pouvoir visualiser une stratégie qu'il a créer.\\
	- Joueur: Veut pouvoir visualiser une stratégie à apprendre.\\
	\\
	\textbf{Principal scénario de succès :}\\
	1. Entraineur ou Joueur démarre le processus de visualisation de la stratégie.\\
	2. Système calcul la position des éléments et les affiches.\\
	\hspace{1cm} \em Système répète l'action 2 jusqu'à la fin de la stratégie.
	\\
	\textbf{Extensions :}\\
	\\
	\textbf{Fréquence d'occurrence :} Régulièrement\\
	\\
	\hline
\end{tabular}

\newpage
\begin{flushleft}
	\textbf{Cas d'utilisation 6 : Placer les éléments}\\
\end{flushleft}
\begin{tabular}{|p{16cm}|}
	\hline
	\\
	\textbf{Projet :} Visualigue\\
	\\
	\textbf{Niveau :} But d'utilisateur\\
	\\
	\textbf{Acteurs primaires :} Entraineur\\
	\\
	\textbf{Figurants et intérêts :} \\
	- Entraineur: Veut pouvoir placer les éléments sur la stratégie et les modifier à sa guise.\\
	\\
	\textbf{Postconditions :} Les éléments sont placés selon ce que l'entraineur souhaite.\\
	\\
	\textbf{Principal scénario de succès :}\\
	1. Entraineur sélectionne un élément de la liste d'éléments disponibles.\\
	2. Système crée un élément selon les spécifications définies dans le sport.\\
	3. Entraineur clique dans l'aire de jeu à l'endroit où l'élément doit être placé.\\
	4. Système place l'élément sur l'aire de jeu et affiche l'élément.\\
	5. Système met à jour la disponibilité de l'élément dans la liste d'éléments disponibles (si nécessaire)\\
	\\
	\textbf{Extensions :}\\
	*a. Entraineur annule le placement de l'élément.\\
	\hspace{0.5cm}1. Système demande à Entraineur s'il est certain de vouloir annuler le placement de l'élément.\\
	\hspace{0.5cm}2. Entraineur confirme qu'il veut bien annuler le placement de l'élément.\\
	\hspace{1cm}2a. Entraineur indique qu'il ne veut plus annuler le placement de l'élément.\\
	\hspace{1.5cm}1. Système revient où il était dans le processus de placement d'un élément.\\
	\hspace{0.5cm}3. Système supprime l'élément créé et réinitialise sa disponibilité (si nécessaire) dans la liste des éléments.\\
	5a. Entraineur souhaite modifier la position de l'élément après l'avoir placé.\\
	\hspace{0.5cm}1. Entraineur clique sur l'élément à modifier.\\
	\hspace{0.5cm}2. Système indique visuellement que l'élément est sélectionné et charge la fenêtre Propriétés avec les paramètres associés.\\
	\hspace{0.5cm}3. Entraineur déplace avec la souris l'élément en question ou modifie la propriété "Position" des paramètres.\\
	\hspace{0.5cm}4. Système déplace l'élément selon les spécifications de l'Entraineur.\\
	5b. Entraineur souhaite modifier la rotation de l'élément.\\
	\hspace{0.5cm}1. Entraineur clique sur l'élément à modifier.\\
	\hspace{0.5cm}2. Système indique visuellement que l'élément est sélectionné et charge la fenêtre Propriétés avec les paramètres associés.\\
	\hspace{0.5cm}3. Système affiche une flèche de rotation près de l'élément sélectionné.\\
	\hspace{0.5cm}4. Entraineur déplace la flèche de rotation jusqu'à l'angle souhaité ou modifie la propriété "Rotation" des paramètres.\\
	\hspace{0.5cm}5. Système oriente l'élément selon les spécifications de l'Entraineur.\\
	5c. Entraineur souhaite modifier le rôle d'un l'élément de type "Joueur".\\
	\hspace{0.5cm}1. Entraineur clique sur l'élément à modifier.\\
	\hspace{0.5cm}2. Système indique visuellement que l'élément est sélectionné et charge la fenêtre Propriétés avec les paramètres associés.\\
	\hspace{0.5cm}3. Entraineur sélectionne le rôle du joueur à partir d'une liste déroulante dans la fenêtre Propriétés.\\
	\hspace{0.5cm}4. Système modifie l'élément pour correspondre aux spécifications du rôle du joueur.\\
	\\
	\textbf{Fréquence d'occurrence :} Régulièrement\\
	\\
	\hline
\end{tabular}

\newpage
\begin{flushleft}
	\textbf{Cas d'utilisation 9 : Exporter une stratégie en format image}\\
\end{flushleft}
\begin{tabular}{|p{16cm}|}
	\hline
	\\
	\textbf{Projet :} Visualigue\\
	\\
	\textbf{Niveau :} But d'utilisateur\\
	\\
	\textbf{Acteurs primaires :} Entraineur\\
	\\
	\textbf{Figurants et intérêts :} \\
	- Entraineur: Veut pouvoir exporter les fichiers de l'applications en image.\\
	\\
	\textbf{Principal scénario de succès :}\\
	1. Entraineur démarre le processus d'exportation.\\
	2. Système demande en quel format les fichiers doivent être exporter.\\
	3. Entraineur sélectionne un format d'image.\\
	4. Système enregistre la stratégie sous forme d'image.\\
	\\
	\textbf{Extensions :}\\
	\\
	\textbf{Fréquence d'occurrence :} Parfois\\
	\\
	\hline
\end{tabular}

\newpage
\begin{flushleft}
	\textbf{Cas d'utilisation 2 : Sauvegarder une stratégie}\\
\end{flushleft}
\begin{tabular}{|p{16cm}|}
	\hline
	\\
	\textbf{Projet :} Visualigue\\
	\\
	\textbf{Niveau :} But d'utilisateur\\
	\\
	\textbf{Acteurs primaires :} Entraineur\\
	\\
	\textbf{Figurants et intérêts :} \\
	- Entraineur: Veut pouvoir sauvegarder une stratégie qu'il a élaborée.\\
	\\
	\textbf{Postconditions :} La stratégie est sauvegardée et peut être chargée lors d'une prochaine utilisation du logiciel.\\
	\\
	\textbf{Principal scénario de succès :}\\
	1. Entraineur démarre le processus de sauvegarde de la stratégie.\\
	2. Système sauvegarde la stratégie dans le fichier de stratégie.\\
	\\
	\textbf{Extensions :}\\
	*a. Entraineur annule le processus de sauvegarde d'une stratégie.\\
	\hspace{1cm}1. Système demande à Entraineur s'il est certain de vouloir annuler la sauvegarde d'une stratégie.\\
	\hspace{1cm}2. Entraineur confirme qu'il veut bien annuler la sauvegarde d'une stratégie.\\
	\hspace{2cm}2a. Entraineur indique qu'il ne veut plus annuler la sauvegarde d'une stratégie.\\
	\hspace{3cm}1. Système revient où il était dans le processus de sauvegarde d'une stratégie.\\
	\hspace{1cm}3. Système retourne à la page où il se trouvait avant que le processus de sauvegarde d'une stratégie soit démarré.\\
	\\
	\textbf{Fréquence d'occurrence :} Régulièrement\\
	\\
	\hline
\end{tabular}


\newpage
\begin{flushleft}
	\textbf{Cas d'utilisation 7 : Configurer les types de sports}\\
\end{flushleft}
\begin{tabular}{|p{16cm}|}
	\hline
	\\
	\textbf{Projet :} Visualigue\\
	\\
	\textbf{Niveau :} But d'utilisateur\\
	\\
	\textbf{Acteurs primaires :} Entraineur\\
	\\
	\textbf{Figurants et intérêts :} \\
	- Entraineur: Veut pouvoir créer et modifier les types de sports supporter par le logiciel.\\
	\\
	\textbf{Postconditions :} Une fois qu'un type de sport est créé, il est possible de créer une stratégie pour ce type de sport et de configurer les obstacles associés à ce type de sport.\\
	\\
	\textbf{Garanties en cas de succès :} Le type de sport a été enregistré.\\
	\\
	\textbf{Principal scénario de succès :}\\
	1. Entraineur démarre le processus de configuration d'un  type de sport.\\
	2. Système demande lequel des types de sport l'utilisateur veut configurer ou s'il veut en créer un nouveau type de sport.\\
	3. Entraineur indique le type à configurer.\\
	4. Système offre de modifier les spécifications du type de sport, telle que le nom du sport, l'imgae pour représenter le terrain, les dimensions réelles du terrain, le nombre de joueurs et les catégories de joueurs associées au sport.\\
	5. Entraineur modifie des spécifications du type de sport.\\
	6. Entraineur démarre le processus d'enregistrement.\\
	7. Système valide et enregistre les modifications du type de sport.\\
	\\
	\textbf{Extensions :}\\
	*a. Entraineur annule le processus de configuration du type de sport.\\
	\hspace{1cm}1. Système demande à Entraineur s'il est certain de vouloir annuler la configuration du type de sport.\\
	\hspace{1cm}2. Entraineur confirme qu'il veut bien annuler la configuration du type de sport.\\
	\hspace{2cm}2a. Entraineur indique qu'il ne veut plus annuler la configuration du type de sport.\\
	\hspace{3cm}1. Système revient où il était dans le processus de configuration.\\
	\hspace{1cm}3. Système retourne à la page où il se trouvait avant que le processus de configuration ne soit démarré.\\
	3a. Entraineur indique qu'il veut créer un nouveau type de sport.\\
	\hspace{1cm}1. Système demande à Entraineur de remplir tout les champs de spécification du type de sport.\\
	\hspace{1cm}2. Entraineur entre toutes les spécifications du type de sport.\\
	\hspace{1cm}3. Entraineur démarre le processus d'enregistrement.\\
	\hspace{1cm}4. Système valide et enregistre la création du type de sport.\\
	7a. Système valide les modifications et constate des données invalides.\\
	\hspace{1cm}1. Système affichae un message d'erreur à l'utlisateur et demande de corriger les données invalides.\\
	\hspace{1cm}2. Entraineur corrige les données.\\
	\\
	\textbf{Fréquence d'occurrence :} Parfois\\
	\\
	\hline
\end{tabular}

\newpage
\begin{flushleft}
	\textbf{Cas d'utilisation 8 : Configurer les obstacles}\\
\end{flushleft}
\begin{tabular}{|p{16cm}|}
	\hline
	\\
	\textbf{Projet :} Visualigue\\
	\\
	\textbf{Niveau :} But d'utilisateur\\
	\\
	\textbf{Acteurs primaires :} Entraineur\\
	\\
	\textbf{Figurants et intérêts :} \\
	- Entraineur: Veut pouvoir créer et modifier les types de sports supporter par le logiciel.\\
	\\
	\textbf{Préconditions :} Un type de sport a été créé.\\
	\\
	\textbf{Garanties en cas de succès :} Les obstacles ont été enregistré.\\
	\\
	\textbf{Principal scénario de succès :}\\
	1. Entraineur démarre le processus de configuration des obstacles.\\
	2. Système demande pour lequel des types de sport l'utilisateur veut configurer les obstacles.\\
	3. Entraineur indique le type désiré.\\
	4. Système offre d'ajouter, de modifier ou de supprimer un obsatce.\\
	5. Entraineur choisi de modifier les spécifications d'un obstacle.\\
	6. Système offre de modifier les spécifications de l'obstacle, telle que le nom de l'obstacle, l'imgae pour représenter l'obstacle et les dimensions réelles de l'obstacle.\\
	7. Entraineur modifie des spécifications de l'obstacle.\\
	8. Entraineur démarre le processus d'enregistrement.\\
	9. Système valide et enregistre les modifications de l'obstacle.\\
	\\
	\textbf{Extensions :}\\
	*a. Entraineur annule le processus de configuration de l'obstacle.\\
	\hspace{1cm}1. Système demande à Entraineur s'il est certain de vouloir annuler la configuration de l'obstacle.\\
	\hspace{1cm}2. Entraineur confirme qu'il veut bien annuler la configuration de l'obstacle.\\
	\hspace{2cm}2a. Entraineur indique qu'il ne veut plus annuler la configuration de l'obstacle.\\
	\hspace{3cm}1. Système revient où il était dans le processus de configuration.\\
	\hspace{1cm}3. Système retourne à la page où il se trouvait avant que le processus de configuration ne soit démarré.\\
	5a. Entraineur indique qu'il veut créer un nouvelle obstacle.\\
	\hspace{1cm}1. Système demande à Entraineur de remplir tout les champs de spécification de l'obstacle.\\
	\hspace{1cm}2. Entraineur entre toutes les spécifications de l'obstacle.\\
	\hspace{1cm}3. Entraineur démarre le processus d'enregistrement.\\
	\hspace{1cm}4. Système valide et enregistre la création du type de sport.\\
	5b. Entraineur indique qu'il veut supprimer un obstacle.\\
	\hspace{1cm}1. Système demande à Entraineur de confimer son choix.\\
	\hspace{1cm}2. Entraineur confirme qu'il veut bien supprimer l'obstacle.\\
	\hspace{2cm}2a. Entraineur indique qu'il ne veut plus supprimer l'obstacle.\\
	\hspace{3cm}1. Système revient à l'interface d'ajout/modification/suppression.\\
	\hspace{1cm}3. Système supprime l'obstacle et revient à l'interface d'ajout/modification/suppression.\\
	9a. Système valide les modifications et constate des données invalides.\\
	\hspace{1cm}1. Système affichae un message d'erreur à l'utlisateur et demande de corriger les données invalides.\\
	\hspace{1cm}2. Entraineur corrige les données.\\
	\\
	\textbf{Fréquence d'occurrence :} Parfois\\
	\\
	\hline
\end{tabular}
\chapter{Spécifications supplémentaires}
\label{s:supplementary_specification}
\section{Fonctionnalité}

\subsection{Annuler/Rétablir}
Permet à l'utilisateur d'annuler les dernières actions effectuées durant les modifications apportées à une stratégie. Permet aussi de rétablir les dernières actions qui ont été annulées.

\subsection{Afficher les coordonnées de la sourie}
En tout temps durant la modification d'une stratégie, affiche la position de la souris en unité réelles.

\subsection{Gestion des erreurs}
À tout moment, indique à l'utilisateur la cause des erreurs et la marche a suivre (lorsque nécessaire).



\section{Convivialité}

\subsection{Facteur humain}
Une partie des utilisateurs seront de jeunes enfants avec peu d'expérience avec ce genre de système. Ainsi, l'interface de visualisation de stratégie sera conçu pour être intutive afin de facilité l'utilisation de l'application pour les jeunes joueurs. Pour se faire, les fonctionnalités seront clairement affichées et expliquées au besoin.



\section{Supportabilité}

\subsection{Adaptabilité}
Une telle application peut s'avéré utile pour plus de sport que seulement le hockey. Ainsi, il est possible de confirgurer les types de sport et les obstacles supporter par l'application afin de s'adapté à tout les sports et à tout les types d'entrainnements.









\chapter{Modèle du domaine}
\label{s:domain_model}
\chapter{Mock-up}
\label{s:mockup}


\includegraphics[scale=0.55]{mockup/mockup.png}

\section{Description}

\subsection{Section du centre}

Cette section est celle où la scène est affiché. Dans le fond de la scène, s'affiche le terrain. C'est dans cette section où les éléments sont placés. En plaçant la sourie sur un élément, un icône s'affiche. En appuyant dessus, il est possible de modifier l'angle de l'élément.

\subsection{Section de droite}

Cette section contient les paramètres de l'élément actuellement sélectionné. Le "Joueur 1" est le nom de l'élément. C'est un champ texte modifiable. Le "Attaquant" est le rôle du joueur. C'est un comboBox dans lequel les valeurs sont celles entrée dans les paramètres du sport. La position en x et en y de l'élément est modifiable via des champs numériques modifiables. L'angle aussi modifiable en utilisant un champ numérique dont la valeur est restreinte entre -360 et 360. Les valeurs négatives sont converties en valeurs positives.

\subsection{Section du bas}

Cette section permet de gérer la visualisation de la stratégie. Le cercle rouge permet de démarrer l'enregistrement de la stratégie. Les autres boutons sont les boutons traditionnels lors du visionnement de vidéo. Le x2 permet de modifier la vitesse de défilement lors d'avance rapide et de recule. 

En dessous se trouve une ligne du temps. Le rectangle noir, le curseur, indique quel image est affiché dans la scène. En le modifiant, l'image affiché dans la scène change. 

\subsection{Section de gauche}

Cette section contient les boutons permettant d'ajouter des éléments à la scène. C'est la barre d'outil. Le premier à partir du haut est celui utilisé pour déplacer les éléments. Le second sert à créer les éléments mobiles. Le troisième sert à créer les éléments statiques. En maintenant un clic sur le second et le troisième bouton, des icônes apparaissent à droite. Ils permettent de modifier l'élément qui sera ajouter à la scène grâce aux outils d'ajout d'éléments. Le quatrième bouton sert à zoomer sur la scène. Juste en dessous se trouve un champ numérique, permettant de modifier le zoom. 
\chapter{Échéancier}
\label{s:echeancier}

La figure \ref{fig:gantt} présente l'échéancier actuel du projet sous la forme d'un diagramme de Gantt. Les principales activités ainsi que les livrables y sont identifiés.

\begin{figure}[p]
	\centering
	\includegraphics[scale=0.35, angle=90]{fig/gantt.png}
	\caption{Échéancier du projet}
	\label{fig:gantt}
\end{figure}

% Appendices
\appendix
\chapter{Glossaire}
\label{s:glossaire}

Le glossaire présente toutes les définitions des termes simplificateurs utilisés dans le rapport. Il sert à réduire l'ambiguïté du texte.\\

{\def\arraystretch{1.5}\tabcolsep=5pt
\begin{tabularx}{\textwidth}{|l|X|}
	\hline
	Nom & Définition \\
	\hline
	Scène 				& Section principale de l'écran. C'est dans cette section que l'aire de jeu et les éléments sont affichés. \\
	Stratégie  			& Ensemble des joueurs et de leurs déplacements dans la scène. \\
	Élément 			& Objet qui peut être placé dans la scène. \\
	Élément statique 	& Élément qui ne se déplace jamais dans la scène. \\
	Élément mobile 		& Élément qui peut se déplacer dans la scène (ex~: un joueur ou une rondelle). \\
	Joueur				& Élément mobile représentant un membre d'une équipe sportive. \\
	Type de sport		& Ensemble de propriétés définissant un sport telles que les dimensions du terrain ou le nombre de joueurs. \\
	Curseur 			& Élément visuel se déplaçant horizontalement sur la ligne du temps indiquant l'image actuellement affichée sur la scène. \\
	Ligne du temps 		& Section de l'écran où l'on retrouve une représentation visuelle de la progression temporelle de la stratégie. \\
	Image				& Rendu de la scène à un moment donné dans le temps. \\
	Barre d'outils 		& Section de l'écran où sont situés les boutons permettant de placer ou déplacer les éléments. \\
	Balle               & Élément mobile d'un sport qui est l'objet principal du sport. Par exemple, un ballon, une rondelle, une pierre de curling, etc. \\
	\hline
\end{tabularx}
}

\end{document}
