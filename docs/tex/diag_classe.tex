\chapter{Diagramme de classe de conception}
\label{s:diag_classe}

\begin{figure}[H]
	\centering
	\includegraphics[width=\textwidth]{{"fig/diagrams2/Diagramme de classe"}.png}
	\caption{Diagramme de classe de conception avec packages}
\end{figure}
\textbf{Diagramme de classes de conception}\\
Le diagramme de classes de conception est divisé en trois packages : le package UI, le package Domaine et le package IO. Ces trois package suivent le modèle en couches : la couche du UI utilise la couche du domaine, et la couche du domaine, quant à elle, utilise la couche utilitaire (IO dans ce cas-ci). Il n’y a donc pas de communications directes entre les couches inférieures et les couches supérieures. De plus, il y a le moins possible de communications entre les couches supérieures et les couches qui ne sont pas directement en-dessous d’elles.\\

\textbf{UI}\\
Le package du UI contient toutes les classes qui servent à faire l’affichage de l’interface graphique. Ce package est composé d’une classe servant de fenêtre principale, nommée StrategyEditionWindow et de classes servant à faire l’affichage des différents menus de configuration. Ces dernières se nomment StrategyCreationDialog, SportEditionDialog et ElementDescritpionEditionDialog. De plus, ce package est aussi composé de deux classes métiers, servant à faciliter l’affichage, qui sont nommées UIElement et Camera. Finalement, il y a une énumération nommée Toolbox, qui contient tous les outils qu’il est possible de sélectionner dans l’interface graphique. Il est possible de voir dans le diagramme qu’à partir de la classe SportEditionDialog, il sera possible d’accéder à la classe ElementDescritpionEditionDialog. Cela est dû au fait que les éléments sont classés par sport : il est donc primordial de sélectionner le sport, et, par la suite, de sélectionner l’élément à modifier dans le sport. Les différents éléments d’un sport qui peuvent être modifiés sont les rôles des joueurs, les obstacles et la ou les balles de ce sport. Au moment de l’ouverture de ElementDescritpionEditionDialog, selon l’élément sélectionné dans l’interface graphique, on pourra modifier l’élément voulu. Il sera aussi possible d’ajouter l’élément voulu dans le sport sélectionné.\\

\textbf{Domaine}\\
Le package du domaine contient toutes les classes qui font partie de la logique de l’application. Ce package est composé de classes métiers, comme les classes Element, MobileElement, Obstacle, Player, Ball, Trajectory, ElementDescription, ObstacleDescription, PlayerDescription, BallDescription, Sport et Strategy. Ce package est aussi composé d’un contrôleur, qui, dans le cas présent, est nommé GodController. Finalement, il y a une interface servant à faire le pont entre le package du domaine et le package du UI, qui s’appelle Updatable. Les principes qui ont été utilisés dans ce package sont les suivants :\\
\\•	Le contrôleur : Dans notre cas, c’est la classe GodController qui joue le rôle de contrôleur. Elle permet aux autres packages d’utiliser le package du domaine. C’est la seule classe qui est utilisée directement par les autres packages. Elle permet de relayer les appels de l’extérieur aux bons objets du domaine.\\
\\•	L’indirection et la protection contre les variations : Dans notre cas, l’interface Updatable permet de faire des appels à l’interface graphique, tout en n’ayant aucune idée de comment cette dernière est faite. Pour pouvoir fonctionner correctement, la classe de la fenêtre n’a qu’à implémenter cette interface. Cela permet d’immuniser notre package du domaine des changements pouvant survenir dans le package du UI.\\
\\•	Le polymorphisme : Dans le package du domaine, il y a de l’héritage partout où c’est possible. Cela permet d’utiliser le plus possible de polymorphisme. Effectivement, le principal exemple d’héritage dans notre programme est l’arborescence des Element. Effectivement, à la racine de l’arborescence, il y a la classe Element, dont les classes MobileElement et Obstacle héritent. Ensuite, les classes Player et Ball héritent de la classe MobileElement. À chaque niveau de l’arborescence, des méthodes de plus en plus spécifiques sont ajoutées aux classes.\\
\\•	Les autres principes : De plus, nous essayons le plus possible d’utiliser les autres principes dans le package du domaine. En effet, dès que cela est applicable, nous essayons d’utiliser les principes d’expert en information, de créateur, de forte cohésion et de faible couplage.\\

\textbf{IO}\\
Le package IO est un petit package qui ne sert qu’à enregistrer différentes choses dans des fichiers ou à charger différentes choses à partir de fichiers. Ce package est composé de deux classes servant à l’enregistrement, au chargement, à l’importation et à l’exportation de fichiers : il s’agit des classes StrategyIO et SportIO. Puisque les méthodes de ces classes sont statiques, il est possible de les utiliser sans avoir besoin de créer d’instance. C’est pourquoi il n’y a pas de relation de composition ou d’agrégation avec ces classes.
