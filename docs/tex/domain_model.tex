\chapter{Modèle du domaine}
\label{s:domain_model}

Cette section présente le modèle du domaine de VisuaLigue. La figure \ref{fig:domain_model} présente une vue globale de l'application.

\begin{figure}[H]
	\centering
	\includegraphics[width=\textwidth]{{"fig/diagrams/Modele du domaine"}.png}
	\caption{Diagramme du modèle du domaine}
	\label{fig:domain_model}
\end{figure}

Le point d'entrée est l'utilisateur. Celui-ci peut être un Entraineur, un Joueur ou un Parent. Dans ce diagramme, on prend en considération que l'utilisateur est un Entraineur, car il peut exécuter toutes les actions.

L'utilisateur peut configurer des sports. Un sport est constitué d'un terrain, de plusieurs descriptions de joueurs, de balles et d'obstacles. Le terrain contient l'image, la taille et les dimensions. Il ne peut y avoir qu'un seul terrain par sport. Les descriptions des joueurs, des balles et des obstacles contiennent les propriétés qui seront utilisées lors de l'ajout de ceux-ci dans une stratégie. Par exemple, c'est ici que l'on définit le rôle des joueurs. Puisque plusieurs propriétés sont partagées entre ces classes, une généralisation appelée \textit{DescriptionÉlément} a été créée. Afin d'alléger le diagramme, la classe utilisateur n'a pas été reliée aux descriptions ni au terrain. Toutefois, on comprendra que lorsque l'utilisateur configure un sport, il modifie aussi les descriptions et le terrain.  Il est aussi à noter que le terme « balle » est utilisé pour décrire tout élément mobile principal d'un sport. Par exemple, il peut représenter une rondelle de hockey, un Frisbee ou même une pierre de curling. Voir le glossaire à la page \pageref{s:glossaire} pour plus de détails.

Une fois que l'utilisateur a configuré un sport, il peut créer et modifier une stratégie. Chaque stratégie est reliée à un sport qui représente les règles, le terrain ainsi que les descriptions des éléments. À l'intérieur d'une stratégie, on retrouve des obstacles, des balles et des joueurs. Ceux-ci partagent des propriétés en commun, d'où la généralisation \textit{Élément}. Chaque élément est relié à sa description afin de refléter dans la stratégie les changements apportés dans le sport. Pour les éléments mobiles, une classe Trajectoire permet d'enregistrer leur déplacement. Encore une fois, pour alléger le diagramme, l'utilisateur n'a pas été relié aux éléments. On comprendra que lorsque l'on modifie une stratégie, on modifie également les éléments qui la composent.