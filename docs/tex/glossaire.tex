\chapter{Glossaire}
\label{s:glossaire}

Le glossaire présente toutes les définitions des termes simplificateurs utilisés dans le rapport. Il sert à réduire l'ambiguïté du texte.\\

{\def\arraystretch{1.5}\tabcolsep=5pt
\begin{tabularx}{\textwidth}{|l|X|}
	\hline
	Nom & Définition \\
	\hline
	Scène 				& Section principale de l'écran. C'est dans cette section que l'aire de jeu et les éléments sont affichés. \\
	Stratégie  			& Ensemble des joueurs et de leurs déplacements dans la scène. \\
	Élément 			& Objet qui peut être placé dans la scène. \\
	Élément statique 	& Élément qui ne se déplace jamais dans la scène. \\
	Élément mobile 		& Élément qui peut se déplacer dans la scène (ex.~: un joueur ou une rondelle). \\
	Joueur				& Élément mobile représentant un membre d'une équipe sportive. \\
	Type de sport		& Ensemble de propriétés définissant un sport telles que les dimensions du terrain ou le nombre de joueurs. \\
	Curseur 			& Élément visuel se déplaçant horizontalement sur la ligne du temps indiquant l'image actuellement affichée sur la scène. \\
	Ligne du temps 		& Section de l'écran où l'on retrouve une représentation visuelle de la progression temporelle de la stratégie. \\
	Image				& Rendu de la scène à un moment donné dans le temps. \\
	Barre d'outils 		& Section de l'écran où sont situés les boutons permettant de placer ou déplacer les éléments. \\
	Balle               & Élément mobile d'un sport qui est l'objet principal du sport. Par exemple, un ballon, une rondelle, une pierre de curling, etc. \\
	\hline
\end{tabularx}
}