\chapter{Modèle des cas d'utilisation}
\label{s:use_cases}
\newpage
\begin{flushleft}
	\textbf{Cas d'utilisation 1 : Créer une stratégie}\\
\end{flushleft}
\begin{tabular}{|p{16cm}|}
	\hline
	\\
	\textbf{Projet :} Visualigue\\
	\\
	\textbf{Niveau :} But d'utilisateur\\
	\\
	\textbf{Acteurs primaires :} Entraineur\\
	\\
	\textbf{Figurants et intérêts :} \\
	- Entraineur: Veut pouvoir créer des fichiers qui contiendront éventuellement des stratégies.\\
	\\
	\textbf{Postconditions :} L'entraineur aura un fichier qui pourra être utilisé pour élaborer une stratégie.\\
	\\
	\textbf{Principal scénario de succès :}\\
	1. Entraineur démarre le processus de création d'une stratégie.\\
	2. Système demande les informations en lien avec la nouvelle stratégie.\\
	3. Entraineur fourni à Système les informations nécessaires.\\
	4. Système demande l'endroit où le fichier devra être enregistré ainsi que le nom de ce dernier.\\
	5. Entraineur choisi l'endroit où le fichier devra être enregistré ainsi que son nom.\\
	6. Système crée le fichier et il est possible de le modifier.\\
	\\
	\textbf{Extensions :}\\
	*a. Entraineur annule le processus de création d'une stratégie.\\
	\hspace{1cm}1. Système demande à Entraineur s'il est certain de vouloir annuler la création d'une stratégie.\\
	\hspace{1cm}2. Entraineur confirme qu'il veut bien annuler la création d'une stratégie.\\
	\hspace{2cm}2a. Entraineur indique qu'il ne veut plus annuler la création d'une stratégie.\\
	\hspace{3cm}1. Système revient où il était dans le processus de création d'une stratégie.\\
	\hspace{1cm}3. Système retourne à la page où il se trouvait avant que le processus de création d'une stratégie ne soit démarré.\\
	1a. Entraineur était au milieu de l'édition d'un fichier.\\
	\hspace{1cm}1. Système demande à Entraineur s'il veut sauvegarder le fichier dont l'édition était en cours.\\
	\hspace{1cm}2. Entraineur choisi s'il veut sauvegarder ou non le fichier dont l'édition était en cours.\\
	5a. Entraineur entre un nom de fichier invalide.\\
	\hspace{1cm}1. Système informe Entraineur que le nom entré est invalide.\\
	\hspace{1cm}2. Entraineur entre un nom valide pour le fichier.\\
	5b. Entraineur choisi un emplacement invalide pour le fichier.\\
	\hspace{1cm}1. Système informe Entraineur que l'emplacement choisi est invalide.\\
	\hspace{1cm}2. Entraineur choisi un emplacement valide pour le fichier.\\
	\\
	\textbf{Fréquence d'occurrence :} Régulièrement\\
	\\
	\textbf{Questions ouvertes :} Quelles seront les informations en lien avec la nouvelle stratégie que l'entraineur devra entrer?\\
	\\
	\hline
\end{tabular}
\newpage
\begin{flushleft}
	\textbf{Cas d'utilisation 2 : Sauvegarder une stratégie}\\
\end{flushleft}
\begin{tabular}{|p{16cm}|}
	\hline
	\\
	\textbf{Projet :} Visualigue\\
	\\
	\textbf{Niveau :} But d'utilisateur\\
	\\
	\textbf{Acteurs primaires :} Entraineur\\
	\\
	\textbf{Figurants et intérêts :} \\
	- Entraineur: Veut pouvoir sauvegarder une stratégie qu'il a élaborée.\\
	\\
	\textbf{Postconditions :} La stratégie est sauvegardée et peut être chargée lors d'une prochaine utilisation du logiciel.\\
	\\
	\textbf{Principal scénario de succès :}\\
	1. Entraineur démarre le processus de sauvegarde de la stratégie.\\
	2. Système sauvegarde la stratégie dans le fichier de stratégie.\\
	\\
	\textbf{Extensions :}\\
	*a. Entraineur annule le processus de sauvegarde d'une stratégie.\\
	\hspace{1cm}1. Système demande à Entraineur s'il est certain de vouloir annuler la sauvegarde d'une stratégie.\\
	\hspace{1cm}2. Entraineur confirme qu'il veut bien annuler la sauvegarde d'une stratégie.\\
	\hspace{2cm}2a. Entraineur indique qu'il ne veut plus annuler la sauvegarde d'une stratégie.\\
	\hspace{3cm}1. Système revient où il était dans le processus de sauvegarde d'une stratégie.\\
	\hspace{1cm}3. Système retourne à la page où il se trouvait avant que le processus de sauvegarde d'une stratégie soit démarré.\\
	\\
	\textbf{Fréquence d'occurrence :} Régulièrement\\
	\\
	\hline
\end{tabular}
\newpage
\begin{flushleft}
	\textbf{Cas d'utilisation 3 : Charger une stratégie}\\
\end{flushleft}
\begin{tabular}{|p{16cm}|}
	\hline
	\\
	\textbf{Projet :} Visualigue\\
	\\
	\textbf{Niveau :} But d'utilisateur\\
	\\
	\textbf{Acteurs primaires :} Entraineur et Joueur\\
	\\
	\textbf{Figurants et intérêts :} \\
	- Entraineur: Veut pouvoir charger une stratégie qu'il a élaborée.\\
	- Joueur: Veut pouvoir charger une stratégie élaborée par l'entraineur.\\
	\\
	\textbf{Postconditions :} La stratégie est chargée et elle peut être modifiée ou visualisée.\\
	\\
	\textbf{Principal scénario de succès :}\\
	1. Entraineur ou Joueur démarre le processus de chargement de la stratégie.\\
	2. Système demande lequel des fichiers de stratégie charger.\\
	3. Entraineur ou Joueur choisi lequel des fichiers de stratégie charger.\\
	4. Système charge le fichier.\\
	\\
	\textbf{Extensions :}\\
	*a. Entraineur ou Joueur annule le processus de chargement d'une stratégie.\\
	\hspace{1cm}1. Système demande à Entraineur ou Joueur s'il est certain de vouloir annuler le chargement d'une stratégie.\\
	\hspace{1cm}2. Entraineur ou Joueur confirme qu'il veut bien annuler le chargement d'une stratégie.\\
	\hspace{2cm}2a. Entraineur ou Joueur indique qu'il ne veut plus annuler le chargement d'une stratégie.\\
	\hspace{3cm}1. Système revient où il était dans le processus de chargement d'une stratégie.\\
	\hspace{1cm}3. Système retourne à la page où il se trouvait avant que le processus de chargement d'une stratégie ne soit démarré.\\
	1a. Entraineur était au milieu de l'édition d'un fichier.\\
	\hspace{1cm}1. Système demande à Entraineur s'il veut sauvegarder le fichier dont l'édition était en cours.\\
	\hspace{1cm}2. Entraineur choisi s'il veut sauvegarder ou non le fichier dont l'édition était en cours.\\
	3a. Entraineur ou Joueur choisi un fichier invalide.\\
	\hspace{1cm}1. Système informe Entraineur ou Joueur que le fichier choisi est invalide.\\
	\hspace{1cm}2. Entraineur ou Joueur choisi un fichier valide.\\
	\\
	\textbf{Fréquence d'occurrence :} Régulièrement\\
	\\
	\hline
\end{tabular}

\newpage
\begin{flushleft}
	\textbf{Cas d'utilisation 4 : Visualiser une stratégie}\\
\end{flushleft}
\begin{tabular}{|p{16cm}|}
	\hline
	\\
	\textbf{Projet :} Visualigue\\
	\\
	\textbf{Niveau :} But d'utilisateur\\
	\\
	\textbf{Acteurs primaires :} Entraineur et Joueur\\
	\\
	\textbf{Figurants et intérêts :} \\
	- Entraineur: Veut pouvoir visualiser une stratégie qu'il a créer.\\
	- Joueur: Veut pouvoir visualiser une stratégie à apprendre.\\
	\\
	\textbf{Principal scénario de succès :}\\
	1. Entraineur ou Joueur démarre le processus de visualisation de la stratégie.\\
	2. Système calcul la position des éléments et les affiches.\\
	\hspace{1cm} \em Système répète l'action 2 jusqu'à la fin de la stratégie.
	\\
	\textbf{Extensions :}\\
	\\
	\textbf{Fréquence d'occurrence :} Régulièrement\\
	\\
	\hline
\end{tabular}

\newpage
\begin{flushleft}
	\textbf{Cas d'utilisation 9 : Exporter une stratégie en format image}\\
\end{flushleft}
\begin{tabular}{|p{16cm}|}
	\hline
	\\
	\textbf{Projet :} Visualigue\\
	\\
	\textbf{Niveau :} But d'utilisateur\\
	\\
	\textbf{Acteurs primaires :} Entraineur\\
	\\
	\textbf{Figurants et intérêts :} \\
	- Entraineur: Veut pouvoir exporter les fichiers de l'applications en image.\\
	\\
	\textbf{Principal scénario de succès :}\\
	1. Entraineur démarre le processus d'exportation.\\
	2. Système demande en quel format les fichiers doivent être exporter.\\
	3. Entraineur sélectionne un format d'image.\\
	4. Système enregistre la stratégie sous forme d'image.\\
	\\
	\textbf{Extensions :}\\
	\\
	\textbf{Fréquence d'occurrence :} Parfois\\
	\\
	\hline
\end{tabular}