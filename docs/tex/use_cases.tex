\chapter{Modèle des cas d'utilisation}
\label{s:use_cases}


\section{Cas d'utilisation : Créer une stratégie}
\begin{longtable}{|p{16cm}|}
	\hline
	\\
	\textbf{Projet :} VisuaLigue\\
	\\
	\textbf{Niveau :} But d'utilisateur\\
	\\
	\textbf{Acteurs primaires :} Entraineur\\
	\\
	\textbf{Figurants et intérêts :} \\
	- Entraineur: Veut pouvoir créer des fichiers qui contiendront éventuellement des stratégies.\\
	\\
	\textbf{Postconditions :} L'entraineur aura un fichier qui pourra être utilisé pour élaborer une stratégie.\\
	\\
	\textbf{Principal scénario de succès :}\\
	1. Entraineur démarre le processus de création d'une stratégie.\\
	2. Système demande les informations en lien avec la nouvelle stratégie.\\
	3. Entraineur fourni à Système les informations nécessaires.\\
	4. Système demande l'endroit où le fichier devra être enregistré ainsi que le nom de ce dernier.\\
	5. Entraineur choisi l'endroit où le fichier devra être enregistré ainsi que son nom.\\
	6. Système crée le fichier et il est possible de le modifier.\\
	\\
	\textbf{Extensions :}\\
	*a. Entraineur annule le processus de création d'une stratégie.\\
	\hspace{1cm}1. Système demande à Entraineur s'il est certain de vouloir annuler la création d'une stratégie.\\
	\hspace{1cm}2. Entraineur confirme qu'il veut bien annuler la création d'une stratégie.\\
	\hspace{2cm}2a. Entraineur indique qu'il ne veut plus annuler la création d'une stratégie.\\
	\hspace{3cm}1. Système revient où il était dans le processus de création d'une stratégie.\\
	\hspace{1cm}3. Système retourne à la page où il se trouvait avant que le processus de création d'une stratégie ne soit démarré.\\
	1a. Entraineur était au milieu de l'édition d'un fichier.\\
	\hspace{1cm}1. Système demande à Entraineur s'il veut sauvegarder le fichier dont l'édition était en cours.\\
	\hspace{1cm}2. Entraineur choisi s'il veut sauvegarder ou non le fichier dont l'édition était en cours.\\
	5a. Entraineur entre un nom de fichier invalide.\\
	\hspace{1cm}1. Système informe Entraineur que le nom entré est invalide.\\
	\hspace{1cm}2. Entraineur entre un nom valide pour le fichier.\\
	5b. Entraineur choisi un emplacement invalide pour le fichier.\\
	\hspace{1cm}1. Système informe Entraineur que l'emplacement choisi est invalide.\\
	\hspace{1cm}2. Entraineur choisi un emplacement valide pour le fichier.\\
	\\
	\textbf{Fréquence d'occurrence :} Régulièrement\\
	\\
	\textbf{Questions ouvertes :} Quelles seront les informations en lien avec la nouvelle stratégie que l'entraineur devra entrer?\\
	\\
	\hline
\end{longtable}

\section{Cas d'utilisation : Sauvegarder une stratégie}
\begin{longtable}{|p{16cm}|}
	\hline
	\\
	\textbf{Projet :} VisuaLigue\\
	\\
	\textbf{Niveau :} But d'utilisateur\\
	\\
	\textbf{Acteurs primaires :} Entraineur\\
	\\
	\textbf{Figurants et intérêts :} \\
	- Entraineur: Veut pouvoir sauvegarder une stratégie qu'il a élaborée.\\
	\\
	\textbf{Postconditions :} La stratégie est sauvegardée et peut être chargée lors d'une prochaine utilisation du logiciel.\\
	\\
	\textbf{Principal scénario de succès :}\\
	1. Entraineur démarre le processus de sauvegarde de la stratégie.\\
	2. Système sauvegarde la stratégie dans le fichier de stratégie.\\
	\\
	\textbf{Extensions :}\\
	*a. Entraineur annule le processus de sauvegarde d'une stratégie.\\
	\hspace{1cm}1. Système demande à Entraineur s'il est certain de vouloir annuler la sauvegarde d'une stratégie.\\
	\hspace{1cm}2. Entraineur confirme qu'il veut bien annuler la sauvegarde d'une stratégie.\\
	\hspace{2cm}2a. Entraineur indique qu'il ne veut plus annuler la sauvegarde d'une stratégie.\\
	\hspace{3cm}1. Système revient où il était dans le processus de sauvegarde d'une stratégie.\\
	\hspace{1cm}3. Système retourne à la page où il se trouvait avant que le processus de sauvegarde d'une stratégie soit démarré.\\
	\\
	\textbf{Fréquence d'occurrence :} Régulièrement\\
	\\
	\hline
\end{longtable}

\section{Cas d'utilisation : Charger une stratégie}
\begin{longtable}{|p{16cm}|}
	\hline
	\\
	\textbf{Projet :} VisuaLigue\\
	\\
	\textbf{Niveau :} But d'utilisateur\\
	\\
	\textbf{Acteurs primaires :} Entraineur et Joueur\\
	\\
	\textbf{Figurants et intérêts :} \\
	- Entraineur: Veut pouvoir charger une stratégie qu'il a élaborée.\\
	- Joueur: Veut pouvoir charger une stratégie élaborée par l'entraineur.\\
	\\
	\textbf{Postconditions :} La stratégie est chargée et elle peut être modifiée ou visualisée.\\
	\\
	\textbf{Principal scénario de succès :}\\
	1. Entraineur ou Joueur démarre le processus de chargement de la stratégie.\\
	2. Système demande lequel des fichiers de stratégie charger.\\
	3. Entraineur ou Joueur choisi lequel des fichiers de stratégie charger.\\
	4. Système charge le fichier.\\
	\\
	\textbf{Extensions :}\\
	*a. Entraineur ou Joueur annule le processus de chargement d'une stratégie.\\
	\hspace{1cm}1. Système demande à Entraineur ou Joueur s'il est certain de vouloir annuler le chargement d'une stratégie.\\
	\hspace{1cm}2. Entraineur ou Joueur confirme qu'il veut bien annuler le chargement d'une stratégie.\\
	\hspace{2cm}2a. Entraineur ou Joueur indique qu'il ne veut plus annuler le chargement d'une stratégie.\\
	\hspace{3cm}1. Système revient où il était dans le processus de chargement d'une stratégie.\\
	\hspace{1cm}3. Système retourne à la page où il se trouvait avant que le processus de chargement d'une stratégie ne soit démarré.\\
	1a. Entraineur était au milieu de l'édition d'un fichier.\\
	\hspace{1cm}1. Système demande à Entraineur s'il veut sauvegarder le fichier dont l'édition était en cours.\\
	\hspace{1cm}2. Entraineur choisi s'il veut sauvegarder ou non le fichier dont l'édition était en cours.\\
	3a. Entraineur ou Joueur choisi un fichier invalide.\\
	\hspace{1cm}1. Système informe Entraineur ou Joueur que le fichier choisi est invalide.\\
	\hspace{1cm}2. Entraineur ou Joueur choisi un fichier valide.\\
	\\
	\textbf{Fréquence d'occurrence :} Régulièrement\\
	\\
	\hline
\end{longtable}

\section{Cas d'utilisation : Visualiser une stratégie}
\begin{longtable}{|p{16cm}|}
	\hline
	\\
	\textbf{Projet :} VisuaLigue\\
	\\
	\textbf{Niveau :} But d'utilisateur\\
	\\
	\textbf{Acteurs primaires :} Entraineur et Joueur\\
	\\
	\textbf{Figurants et intérêts :} \\
	- Entraineur: Veut pouvoir visualiser une stratégie qu'il a créer.\\
	- Joueur: Veut pouvoir visualiser une stratégie à apprendre.\\
	\\
	\textbf{Principal scénario de succès :}\\
	1. Entraineur ou Joueur démarre le processus de visualisation de la stratégie.\\
	2. Système calcul la position des éléments et les affiches.\\
	\hspace{1cm} \em Système répète l'action 2 jusqu'à la fin de la stratégie.
	\\
	\textbf{Extensions :}\\
	\\
	\textbf{Fréquence d'occurrence :} Régulièrement\\
	\\
	\hline
\end{longtable}

\section{Cas d'utilisation : Placer les éléments}
\begin{longtable}{|p{16cm}|}
	\hline
	\\
	\textbf{Projet :} VisuaLigue\\
	\\
	\textbf{Niveau :} But d'utilisateur\\
	\\
	\textbf{Acteurs primaires :} Entraineur\\
	\\
	\textbf{Figurants et intérêts :} \\
	- Entraineur: Veut pouvoir placer les éléments sur la stratégie et les modifier à sa guise.\\
	\\
	\textbf{Postconditions :} Les éléments sont placés selon ce que l'entraineur souhaite.\\
	\\
	\textbf{Principal scénario de succès :}\\
	1. Entraineur sélectionne un élément de la liste d'éléments disponibles.\\
	2. Système crée un élément selon les spécifications définies dans le sport.\\
	3. Entraineur clique dans l'aire de jeu à l'endroit où l'élément doit être placé.\\
	4. Système place l'élément sur l'aire de jeu et affiche l'élément.\\
	5. Système met à jour la disponibilité de l'élément dans la liste d'éléments disponibles (si nécessaire)\\
	\\
	\textbf{Extensions :}\\
	*a. Entraineur annule le placement de l'élément.\\
	\hspace{0.5cm}1. Système demande à Entraineur s'il est certain de vouloir annuler le placement de l'élément.\\
	\hspace{0.5cm}2. Entraineur confirme qu'il veut bien annuler le placement de l'élément.\\
	\hspace{1cm}2a. Entraineur indique qu'il ne veut plus annuler le placement de l'élément.\\
	\hspace{1.5cm}1. Système revient où il était dans le processus de placement d'un élément.\\
	\hspace{0.5cm}3. Système supprime l'élément créé et réinitialise sa disponibilité (si nécessaire) dans la liste des éléments.\\
	5a. Entraineur souhaite modifier la position de l'élément après l'avoir placé.\\
	\hspace{0.5cm}1. Entraineur clique sur l'élément à modifier.\\
	\hspace{0.5cm}2. Système indique visuellement que l'élément est sélectionné et charge la fenêtre Propriétés avec les paramètres associés.\\
	\hspace{0.5cm}3. Entraineur déplace avec la souris l'élément en question ou modifie la propriété "Position" des paramètres.\\
	\hspace{0.5cm}4. Système déplace l'élément selon les spécifications de l'Entraineur.\\
	5b. Entraineur souhaite modifier la rotation de l'élément.\\
	\hspace{0.5cm}1. Entraineur clique sur l'élément à modifier.\\
	\hspace{0.5cm}2. Système indique visuellement que l'élément est sélectionné et charge la fenêtre Propriétés avec les paramètres associés.\\
	\hspace{0.5cm}3. Système affiche une flèche de rotation près de l'élément sélectionné.\\
	\hspace{0.5cm}4. Entraineur déplace la flèche de rotation jusqu'à l'angle souhaité ou modifie la propriété "Rotation" des paramètres.\\
	\hspace{0.5cm}5. Système oriente l'élément selon les spécifications de l'Entraineur.\\
	5c. Entraineur souhaite modifier le rôle d'un l'élément de type "Joueur".\\
	\hspace{0.5cm}1. Entraineur clique sur l'élément à modifier.\\
	\hspace{0.5cm}2. Système indique visuellement que l'élément est sélectionné et charge la fenêtre Propriétés avec les paramètres associés.\\
	\hspace{0.5cm}3. Entraineur sélectionne le rôle du joueur à partir d'une liste déroulante dans la fenêtre Propriétés.\\
	\hspace{0.5cm}4. Système modifie l'élément pour correspondre aux spécifications du rôle du joueur.\\
	5c. Entraineur souhaite supprimer un élément.\\
	\hspace{0.5cm}1. Entraineur clique sur l'élément à supprimer.\\
	\hspace{0.5cm}2. Système indique visuellement que l'élément est sélectionné et charge la fenêtre Propriétés avec les paramètres associés.\\
	\hspace{0.5cm}3. Entraineur clique sur le bouton "Supprimer".\\
	\hspace{0.5cm}4. Système demande à Entraineur s'il veut vraiment supprimer l'élément.\\
	\hspace{0.5cm}5. Entraineur confirme la suppression de l'élément.\\
	\hspace{1cm}5a. Entraineur annule la suppression de l'élément.\\
	\hspace{1.5cm}1. Système revient où il était dans le processus de placement d'un élément.\\
	\hspace{1cm}6. Système supprime l'élément.\\
	\\
	\textbf{Fréquence d'occurrence :} Régulièrement\\
	\\
	\hline
\end{longtable}

\section{Cas d'utilisation : Modifier les trajectoires des éléments mobiles image par image}
\begin{longtable}{|p{16cm}|}
	\hline
	\\
	\textbf{Projet :} VisuaLigue\\
	\\
	\textbf{Niveau :} But d'utilisateur\\
	\\
	\textbf{Acteurs primaires :} Entraineur\\
	\\
	\textbf{Figurants et intérêts :} \\
	- Entraineur: Veut pouvoir modifier la trajectoire des éléments mobiles à sa guise.\\
	\\
	\textbf{Postconditions :} Les trajectoires des éléments mobiles correspondent à ce que l'entraineur souhaite.\\
	\\
	\textbf{Principal scénario de succès :}\\
	1. Entraineur déplace le curseur sur la ligne du temps au début complètement de la séquence ou clique sur le bouton "Retour au début".\\
	2. Système replace les éléments selon leurs positions et orientations d'origine.\\
	3. Entraineur délace les éléments mobiles aux endroits désirés. Voir cas d'utilisation \underline{Placer les éléments}.\\
	4. Entraineur clique sur le bouton "Prochaine image".\\
	5. Système enregistre toutes les positions des éléments mobiles.\\
	6. Système affiche sur la ligne du temps tous les éléments modifiés.\\
	7. Système déplace le curseur de la ligne du temps d'une image.\\
	\textit{Répéter les étapes 3 à 7 jusqu'à la fin de la stratégie.}\\
	\\
	\textbf{Extensions :}\\
	3a. Entraineur souhaite retirer le déplacement d'un élément mobile.\\
	\hspace{0.5cm}1. Entraineur clique sur l'élément mobile désiré.\\
	\hspace{0.5cm}2. Système indique visuellement que l'élément est sélectionné.\\
	\hspace{0.5cm}3. Entraineur clique sur le bouton "Supprimer image-clé".\\
	\hspace{0.5cm}4. Système supprime l'image-clé de l'élément mobile et replace l'élément mobile à son ancien emplacement.\\
	7a. Entraineur souhaite revenir à l'image précédente pour la modifier.\\
	\hspace{0.5cm}1. Entraineur clique sur le bouton "Image précédente" ou clique sur la ligne du temps à l'endroit désiré.\\
	\hspace{0.5cm}2. Système affiche les éléments mobiles à leur position lors de l'image sélectionnée.\\
	\\
	\textbf{Fréquence d'occurrence :} Régulièrement\\
	\\
	\hline
\end{longtable}

\section{Cas d'utilisation : Modifier les trajectoires des éléments mobiles en temps réel}
\begin{longtable}{|p{16cm}|}
	\hline
	\\
	\textbf{Projet :} VisuaLigue\\
	\\
	\textbf{Niveau :} But d'utilisateur\\
	\\
	\textbf{Acteurs primaires :} Entraineur\\
	\\
	\textbf{Figurants et intérêts :} \\
	- Entraineur: Veut pouvoir modifier la trajectoire des éléments mobiles à sa guise et en temps réel.\\
	\\
	\textbf{Postconditions :} Les trajectoires des éléments mobiles correspondent à ce que l'entraineur souhaite.\\
	\\
	\textbf{Principal scénario de succès :}\\
	1. Entraineur déplace le curseur sur la ligne du temps au début complètement de la séquence ou clique sur le bouton "Retour au début".\\
	2. Système replace les éléments selon leurs positions et orientations d'origine.\\
	3. Entraineur clique sur un élément mobile.\\
	4. Système indique visuellement que l'élément est sélectionné.\\
	5. Entraineur clique sur le bouton "Démarrer enregistrement en temps réel".\\
	6. Système affiche un décompte de 3 secondes à l'écran.\\
	7. Système joue la stratégie en temps réel et enregistre les positions de la souris pour chaque image.\\
	8. Système génère une image-clé à chaque image et enregistrent celles-ci pour l'élément mobile sélectionné.\\
	9. Entraineur visualise la stratégie ainsi obtenu selon le cas d'utilisation \underline{Visualiser une stratégie}.\\
	\textit{Répéter les étapes 1 à 9 pour chaque élément mobile.}\\
	\\
	\textbf{Extensions :}\\
	*a. Entraineur souhaite arrêter l'enregistrement.\\
	\hspace{0.5cm}1. Entraineur clique sur le bouton "Annuler l'enregistrement".\\
	\hspace{0.5cm}2. Système arrête l'enregistrement, supprime les images-clés générées et retourne les éléments mobiles à leurs positions initiales.\\
	9a. Entraineur souhaite ré-enregistrer la trajectoire.\\
	\hspace{0.5cm}1. Entraineur répète les étapes 1 à 9 en prenant soin de sélectionner le même élément mobile.\\
	\\
	\textbf{Fréquence d'occurrence :} Régulièrement\\
	\\
	\hline
\end{longtable}

\section{Cas d'utilisation : Exporter une stratégie en format image}
\begin{longtable}{|p{16cm}|}
	\hline
	\\
	\textbf{Projet :} VisuaLigue\\
	\\
	\textbf{Niveau :} But d'utilisateur\\
	\\
	\textbf{Acteurs primaires :} Entraineur\\
	\\
	\textbf{Figurants et intérêts :} \\
	- Entraineur: Veut pouvoir exporter les fichiers de l'applications en image.\\
	\\
	\textbf{Principal scénario de succès :}\\
	1. Entraineur démarre le processus d'exportation.\\
	2. Système demande en quel format les fichiers doivent être exporter.\\
	3. Entraineur sélectionne un format d'image.\\
	4. Système enregistre la stratégie sous forme d'image.\\
	\\
	\textbf{Extensions :}\\
	\\
	\textbf{Fréquence d'occurrence :} Parfois\\
	\\
	\hline
\end{longtable}